%!TEX encoding = UTF-8 Unicode
%!TEX TS-program = pdflatex

%%% --- PREAMBLE --- %%%

\documentclass[a4paper,twocolumn,twoside]{article}

\usepackage[italian]{babel}
\usepackage[T1]{fontenc}
\usepackage[utf8x]{inputenc}
% \renewcommand*\familydefault\sfdefault
\usepackage[margin=1cm,top=1.8cm,inner=1.8cm]{geometry}
\usepackage{amsmath}
\usepackage{amssymb}
\usepackage{hyperref}

\newcommand*\pag{\null\hspace*{0pt}\dotfill}

\frenchspacing

%%% --- DOCUMENT --- %%%

\begin{document}

\title{Checklist meccanica statistica dalle dispense di Rossini}
\date{}

\maketitle
\thispagestyle{empty}

\pagestyle{empty}

\section*{Termodinamica}

\begin{itemize}
	\item def. di trasf. quasi-statica \pag 1
	\item legge zero \pag 3
	\item dimostrare l'esistenza di un'eq. di stato $T=T(p,V)$ partendo dalla legge zero \pag 3
	\item prima legge \pag 3
	\item equazioni trasf. adiabatiche gas ideale \pag 4
	\item seconda legge \pag 4
	\item dimostrare kelvin $\iff$ clausius \pag 5
	\item dimostrare carnot (efficienza) \pag 6
	\item enunciato clausius (entropia) \pag 6
	\item calcolare efficienza ciclo di carnot \pag 6
	\item carnot $\implies$ clausius \pag 7
	\item def. di entropia e proprietà di base \pag 7
	\item clausius $\implies$ carnot \pag 8
	\item terza legge \pag 8
	\item demone di maxwell \pag 8
	\item equazione di eulero \pag 10
	\item relazione di Gibbs-Duhem \pag 10
	\item potenziali termodinamici \pag 10
	\item relazioni di maxwell \pag 11
	\item relazioni di Gibbs-Helmholtz \pag 12
	\item grand-potential \pag 13
	\item principio di le chatelier \pag 13
\end{itemize}

\section*{Meccanica statistica}

\begin{itemize}		
	\item ipotesi ergodica \pag 15
	\item teorema di liouville \pag 15
	\item equiprobabilità a priori \pag 15
\end{itemize}

\subsection*{Microcanonico}

\begin{itemize}
	\item microcanonico \pag 15
	\item def. entropia \pag 16
	\item estensività dell'entropia \pag 16
	\item def. temperatura \pag 17
	\item def. pressione \pag 17
	\item derivazione leggi 1 e 2 \pag 18
	\item teorema di equipartizione \pag 18
	\item relazione tra gradi di libertà e $c_V$ \pag 19
	\item entropia come media di ensemble \pag 19
	\item gas ideale \pag 20
	\item paradosso di Gibbs \pag 21
	\item oscillatori armonici \pag 22
	\item spin in campo magnetico \pag 22
\end{itemize}

\subsection*{Canonico}

\begin{itemize}
	\item ensemble canonico \pag 24
	\item def. energia libera \pag 25
	\item equivalenza con la def. di entropia \pag 25
	\item varianza dell'energia \pag 26
	\item gas ideale \pag 28
	\item distribuzione di Maxwell-Boltzmann \pag 28
	\item spin \pag 29
\end{itemize}

\subsection*{Grancanonico}

\begin{itemize}
	\item ensemble grancanonico \pag 29
	\item gran potenziale \pag 31
	\item varianza del numero di particelle \pag 31
	\item equilibrio chimico \pag 32
	\item gas ideale \pag 34
\end{itemize}

\subsection*{Sistemi interagenti}

\begin{itemize}
	\item equazione di stato non integrata \pag 35
	\item relazione ricorsiva per la funzione di partizione \pag 36
	\item gas ideale \pag 36
	\item gas diluito \pag 36
	\item equazione di van der waals \pag 38
	\item punto critico \pag 38
	\item costruzione di maxwell \pag 39
\end{itemize}

\subsection*{Espansione in cluster}

\begin{itemize}
	\item potenziale di Lennard-Jones \pag 41
	\item funzioni di Mayer \pag 41
	\item espansione in cumulanti \pag 41
	\item relazione fra funzioni di Mayer e cumulanti fino all'ordine~3 \pag 42
	\item relazione generale tra funzioni di Mayer e cumulanti \pag 42
	\item ordine~4 \pag 43
	\item interpretazione come grafi \pag 43
	\item espansione del viriale \pag 44
	\item coefficiente $\ell=2$ \pag 44
	\item coefficiente $N=4$ in termini di grafi \pag 45
	\item coefficiente del viriale in termini di grafi \pag 46
\end{itemize}

\subsection*{Funzioni di correlazione}

\begin{itemize}
	\item densità locale \pag 47
	\item relazione tra energia e densità locale \pag 47
	\item relazione tra pressione e densità locale \pag 47
	\item equazioni di Born-Green \pag 48
	\item approssimazione di Kirkwood \pag 48
\end{itemize}

\subsection*{Transizioni di fase}

\begin{itemize}
	\item classificazione di Ehrenfest \pag 49
	\item definizione di fenomeno critico \pag 49
	\item fatti sperimentali sulle transizioni di fase \pag 50
	\item esponenti critici nell'equazione di Van der Waals \pag 51
	\item modello di Ising \pag 52
	\item integrale gaussiano multidimensionale \pag 52
	\item trasformazione di Hubbard-Stratonovich \pag 53
\end{itemize}

\end{document}