%!TEX encoding = UTF-8 Unicode
%!TEX TS-program = pdflatex

%%% --- PREAMBLE --- %%%

\documentclass[a4paper,twocolumn,twoside]{article}

\usepackage[italian]{babel}
\usepackage[T1]{fontenc}
\usepackage[utf8x]{inputenc}
% \renewcommand*\familydefault\sfdefault
\usepackage[margin=1cm,top=1.8cm,inner=1.8cm]{geometry}
\usepackage{amsmath}
\usepackage{amssymb}
\usepackage{hyperref}

\newcommand*\pag{\null\hspace*{0pt}\dotfill}

\frenchspacing

%%% --- DOCUMENT --- %%%

\begin{document}

\title{Checklist meccanica statistica dalle dispense di Rossini}
\date{}

\maketitle
\thispagestyle{empty}

\section*{Termodinamica}

\begin{itemize}
	\item def. di trasf. quasi-statica \pag 1
	\item legge zero \pag 3
	\item dimostrare l'esistenza di un'eq. di stato $T=T(p,V)$ partendo dalla legge zero \pag 3
	\item prima legge \pag 3
	\item equazioni trasf. adiabatiche gas ideale \pag 4
	\item seconda legge \pag 4
	\item dimostrare kelvin $\iff$ clausius \pag 5
	\item dimostrare carnot (efficienza) \pag 6
	\item enunciato clausius (entropia) \pag 6
	\item calcolare efficienza ciclo di carnot \pag 6
	\item carnot $\implies$ clausius \pag 7
	\item def. di entropia e proprietà di base \pag 7
	\item clausius $\implies$ carnot \pag 8
	\item terza legge \pag 8
	\item demone di maxwell \pag 8
	\item equazione di eulero \pag 10
	\item relazione di Gibbs-Duhem \pag 10
	\item potenziali termodinamici \pag 10
	\item relazioni di maxwell \pag 11
	\item relazioni di Gibbs-Helmholtz \pag 12
	\item grand-potential \pag 13
	\item principio di le chatelier \pag 13
\end{itemize}

\section*{Meccanica statistica}

\begin{itemize}		
	\item ipotesi ergodica \pag 15
	\item teorema di liouville \pag 15
	\item equiprobabilità a priori \pag 15
\end{itemize}

\subsection*{Microcanonico}

\begin{itemize}
	\item microcanonico \pag 15
	\item def. entropia \pag 16
	\item estensività dell'entropia \pag 16
	\item def. temperatura \pag 17
	\item def. pressione \pag 17
	\item derivazione leggi 1 e 2 \pag 18
	\item teorema di equipartizione \pag 18
	\item relazione tra gradi di libertà e $c_V$ \pag 19
	\item entropia come media di ensemble \pag 19
	\item gas ideale \pag 20
	\item paradosso di Gibbs \pag 21
	\item oscillatori armonici \pag 22
	\item spin in campo magnetico \pag 22
\end{itemize}

\subsection*{Canonico}

\begin{itemize}
	\item ensemble canonico \pag 24
	\item def. energia libera \pag 25
	\item equivalenza con la def. di entropia \pag 25
	\item varianza dell'energia \pag 26
	\item gas ideale \pag 28
	\item distribuzione di Maxwell-Boltzmann \pag 28
	\item spin \pag 29
\end{itemize}

\subsection*{Grancanonico}

\begin{itemize}
	\item ensemble grancanonico \pag 29
	\item gran potenziale \pag 31
	\item varianza del numero di particelle \pag 31
	\item equilibrio chimico \pag 32
	\item gas ideale \pag 34
\end{itemize}

\end{document}